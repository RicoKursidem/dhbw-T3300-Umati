\documentclass[a4paper, 12pt, oneside]{scrbook}
\input{settings.tex}
\addbibresource{Bachelorarbeit.bib}

\begin{document}
	\frontmatter
	\def\doctype{Bachelorarbeit}
\def\title{Vorgehensweise zur Implementierung der VDMA Spezifikation „OPC UA for Machinery (OPC 40 001)“ in Maschinen und IT-Systemen erarbeiten}
\def\author{Rico Kursidem}
\def\supervisor{Dipl. Ing (FH) Michalowski}
\def\supervisortwo{Prof. Dr. Mielebacher}

\begin{titlepage}

\vspace{10mm}

\begin{center}
	
	\vspace{5mm}
	\huge \title
	
	\vspace{34pt}
	\large \doctype
		
	\vspace{30pt}	
	\small Angewandte Informatik \\
	\large Duale Hochschule Baden-Württemberg Mosbach \\
	\small Studienpartner \\
	\large AZO GmbH \& Co. KG \\
    \vspace{35pt}
    
    \includegraphics[height=2.5cm]{prefix/image/logo-dhbw.eps}
    \includegraphics[height=2.5cm]{prefix/image/logo-azo.png}
	
	\vspace{40pt}	
	\small von \\
	\large \author \\
	\small betreut von \\
	\large \supervisor \\
	\small und \\
	\large \supervisortwo

\end{center}

\vspace{75pt}


\vspace{49.7pt}

\fancypagestyle{empty}{
  \fancyhf{}
  \fancyfoot[C]{\today}
}

\end{titlepage}
	\chapter*{Abbreviation} 
\begin{acronym}
	%A
	%B
	%C
	%D
	%E
	%F
	%G
	%H
	%I
	\acro{IoT}{Internet of Things}
	%J
	%K
	%L
	%M
	\acro{MES}{Manufacturing Execution System}
	%N
	%O
	\acro{OPC UA}{Open Platform Communications Unified Architecture}
	%P
	%Q
	%R
	%S
	%T
	%U
	\acro{UMATI}{Universal machien technology interface}
	%V
	%W
	%X
	%Y
	%Z
\end{acronym}
	\tableofcontents
	\listoffigures
	\listoftables
	%\lstlistoflistings
	\nocite{*}

	\mainmatter
	
	\chapter*{Abstract}
	
	\section*{Deutsch}
	
	
	\section*{English}
	
	\pagebreak
%	\conclusionpng
	\chapter{Einführung}
	% (4 Seiten)
	
	% Hinführung
	% M2M - Kommunikation
	% vertikale Integration
	% Industrie 4.0
	% added Value to data
	% automated posibilities
	
	\section{Problemstellung und Ziel}
	
	% (1-2 Seiten)
	
	\noindent Durch die zunehmende Automatisierung der Produktion und Fertigung stehen viele Produzenten vor der Aufgabe, zahlreiche Maschinen verschiedenster Hersteller in ihren Fertigungsprozess zu integrieren. Viele dieser Maschinen nutzen verschiedene Kommunikationsprotokolle, Datenformate und Schnittstellen was die Integration dieser diversen Systeme erschwert. Durch eine komplizierte Integration kann Vendor Locking entstehen. Die Bindung eines Kunden an einen Anbieter, da dieser eine Technologie Integriert, die andere Hersteller nicht oder nur extrem Schwer Integrieren können. Dies kann zu einer Atmosphäre führen, welche für den Kosten weniger Flexibilität und Kontrolle über seine eigene Produktion gibt und gleichzeitig die Kosten steigen lässt. Um dieses Problem zu lösen müssen sich die Produzenten zu einem Standard zusammenfinden und die Interoperabilität ihrer Systeme gewährleisten.
	
	\noindent Für AZO als Anlagenproduzent ist es von großem Interesse mit Anlagen und Maschinen anderer Hersteller kommunizieren zu können und dabei die Integration möglich einfach zu realisieren. Auch für Entwickler von \ac(MES) bedeutet die Standardisierung von Kommunikationsprotokollen eine schnellere und einfachere Entwicklung was zu niedrigeren Entwicklungskosten führt und die Robustheit der Systeme erhöhen kann. AZO hat, um diese Standardisierung zu erreichen, an der Ausarbeitung von \ac{UMATI} zugesagt. Dieser Standard möchte die Kommunikation zwischen Maschinen und der darüber liegenden Datenverarbeitung definieren und basiert auf dem Offenen Standard \ac{OPC UA}.
	
	\noindent Das Ziel dieser Arbeit ist es, UMATI in einem Testumfeld aufzubauen und eine Proof of Konzept System aufzubauen. Um dies zu erreichen soll zunächst eine Marktanalyse durchgeführt werden und mögliche Lösungen zum versenden und empfangen von Daten zwischen Maschine und MES zu finden. Diese sollen anhand von selbst entwickelten Kriterien abgewogen und bewertet werden. Die vielversprechendsten Umsetzungen sollen in einem Testsystem implementiert werden und für zukünftige Demonstrationen auf Messen für AZO zur Verfügung gestellt werden. Außerdem soll evaluiert werden, ob UMATI auch in der Low-Code Plattform Node Red angewendet werden kann und wenn möglich, das entstehende System als Open-Source Lösung der Öffentlichkeit zur Verfügung zu stellen.
	
	\noindent Der Prototyp soll ebenfalls eine Integration in ACAS umfassen. Damit kann gezeigt werden, dass UMATI eine Möglichkeit ist, die Kommunikation von AZO Anlagen und ACAS zu realisieren. Diese Integration muss im Nachgang weiter Evaluiert werden um festzustellen ob UMATI als Standardisierung eine gute Lösung ist.
	
	
	
	% AZO -> Entsteht Problem
	% Problemstellung
	% Zielsetzung -> Software die das halt kann
	
	% Markanalyse -> Prototyp -> in ACAS Integrieren
	
	\section{Unternehmen}
	
	%TODO: Welche AZO Bezeichnung
	
	\noindent Diese Arbeit wird mit dem Dualen Partner \textit{AZO Gmbh \& Co. KG} durchgeführt. AZO bietet maßgeschneiderte Lösungen für die automatisierte Förderung, Lagerung und Dosierung von Rohstoffen weltweit an. Dabei werden Anlagen für die Bereiche der Chemie-, Nahrungsmittel-, Pharma-, Kosmetik und Kunststoffindustrie gefertigt. Diese Projekte umfassen die Planung, Fertigung und Montage, sowie die Automatisierung der Anlagen.
	
	\noindent AZO entwickelt ein eigenes MES System ACAS, welches die Kommunikation zwischen Anlagen von AZO, aber auch Anlagen von anderen Herstellern, mit den Kunden vereinheitlichen und verbessern soll. Dabei soll ein Umfassendes System entstehen, welches Steuerung, Visualisierung und Überwachung der Produktion übernehmen und in einer zentralen Software bündeln soll.
	
	\noindent ACAS entsteht in der Entwicklungsabteilung welche ebenfalls an dieser Arbeit beteiligt ist. Sie fokussiert sich auf die Automatisierung von AZO Anlagen im Bereich von SPS Steuerungen bis zur Datenverarbeitung und erhebung auf abstrahierter Ebene. 
	
	%TODO: Was stellt mir AZO zur verfügung - System, Server, ...
	
	\section{Forschungsfragen}
	
	\begin{itemize}
		\item \textbf{Q1}: Welche Herrausforderungen ergeben sich bei der Umsetzung von Umati?
		\item \textbf{Q2}: Ist eine Lösung durch Node Red Möglich? Wenn ja, ist diese Wirtschaftlich oder existieren bereits bessere Lösungen?
		\item \textbf{Q3}: Wie kann die Datenkommunikation von Anlage und ACAS bestmöglich umgesetzt werden?
	\end{itemize}
	
	\section{Aufbau der Arbeit}
	
	% Ziel
	% Methodik - kurz
	% Lesermotivation 
	% Relevanz für AZO, relevanz für Wissenschaft
	% Einsatzbereich bei AZO, Ziel und Anwndung
	
\chapter{Grundlagen}
	% (16 Seiten)
	
	% Stand der Technik
	\section{OPC UA}
	
	\noindent \ac{OPC UA} ist ein Unternehmensunabhängiger, offener Standard zur Kommunikation von Informationen und Daten zwischen Maschinen im Industriellen Umfeld. Das Protokoll wurde 2008 von der OPC Foundation veröffentlicht und nimmt sich zur Aufgabe, die Interoperabilität von Systemen zu fördern. Es verwendet eine Service orientierte Architektur und kann auf verschiedensten Betriebssystemen verwendet werden. OPC UA ist sehr gut Skalierbar und kann deshalb als Kommunikationsprotokoll bei kleinen \ac{IoT} Systemen als auch bei komplexen Cloud Systemen zur Anwendung kommen.
	
	\noindent OPC UA besteht baut auf einem Server-Client-Modell auf, wobei der Client die Anfragen an den Server senden muss. Der Client fragt den Server nach Daten und analysiert diese. Der Server stellt die Daten zur Verfügung. Diese Struktur ermöglicht eine einfache Skalierung von verschiedensten Geräten. 
	
	% OPC UA
		% - Interoperabilität
		
	% Aufbau
	% Aufgaben
	% Standards und Anforderungen
	
	\section{Umati}
	
		% - Automatisierungspyramide
	
		\subsection{VDMA}
		\subsection{Anforderungen}
		\subsection{OPC UA 40001}
		
		% Was ist UMATI (Ziel, Entstehung, Stand der Technik)
		% OPCUA 40001
			% Was wird benötigt zum ...
				% senden
				% empfangen
			% Anmeldung
		
		% Marktanalyse: Bereits vorhandene Lösungen	
	
\chapter{Methodik}
	% (8 Seiten)
	
	
	% Forschungsmethodik - Deduktiv??? - Konstruktiv???
	
	\section{Literaturrecherche}
	% Datenerhebung - Literaturrecherche
	% Stichwort recherche
	
	\section{Evaluation}
	
	\section{Experteninterview}
	
	\section{Zeitplanung}
	
\chapter{Ergebnisse}
	
	% (8 Seiten)
	
	\section{Anforderungsanalyse}
	
	\section{Lösungsansätze}
		% Nach was suche ich, Kriterienentwurf
		% Was bracuh ich Hard Soft
		
		% Literaturrecherche an anderen Implementationen (Umati und 5G)
	
	\section{Marktanalyse}
	
		% Fertige Lösungen
		% Node Red selbstimplementierung
	
	\section{Wirtschaftlichkeit und Projektanalyse}
	
		\subsection{Kostenplanung}
		
		\subsection{Fallbeispiel}
			% Beispielswiese an Fallbeispiel -> Durchschnittliche AZO Straße
			% Interview mit Person die diese Integration macht -> Expertenquellen
			% -> Probleme bei Schnittstellen integrierung.
	
	\section{Implementierung eines Prototyps}
	
		\subsection{Aufbau der Umgebung}
		
		\subsection{Implementierung}
		
			% Proof of Konzept
			% Proof of Konzept mit Node Red
			
			% Probleme und Lösungen
		
		\subsection{Integration in bestehende Infrastruktur}
			% Integration in Unternehmensstruktur
			

	
	
	\chapter{Diskussion und Fazit}
	
		% (2-5 Seiten)	
	
		% Disskussion -> Subjektive Bewertung der Spezifikation UMATI, ... Literaturpunkte meine Meinung, Sinnvoll Einzusetzen
	
	
	\frontmatter
	\printbibliography

\end{document}
