\chapter*{Abstract}
\section*{Deutsch}

In der vorliegenden Arbeit, betitelt \glqq Entwicklung einer Vorgehensweise zur Integration der VDMA-Spezifikation \glq OPC UA for Machinery (OPC 40 001)\grq{} in Maschinen und IT-Systeme\grqq{}, wird der initiierende Schritt zur Implementierung des Umati-Standards untersucht. Das Hauptziel besteht darin, die Funktionalität dieses Standards zu analysieren und eine Methode zur Visualisierung von Daten aus Maschinen mithilfe von Umati zu entwickeln. Die zugrundeliegenden Forschungsfragen lauten: Welche Herausforderungen ergeben sich bei der Integration von Umati? Welche Lösungsansätze sind wirtschaftlich am effizientesten? Wie kann die Kommunikation zwischen Maschinen und Visualisierungsplattformen effektiv umgesetzt werden?

Zur Beantwortung dieser Fragen werden zwei Prototypen konzipiert und realisiert. Der erste Prototyp dient zur Veranschaulichung der Umati-Funktionalität über das von VDMA bereitgestellte Web-Dashboard, während der zweite Prototyp eine Methodik zur Datenübertragung von Maschinen zur Visualisierungsplattform implementiert.

Es wird erkannt, dass Umati in einer Industrie 4.0-Umgebung ein vielversprechender Kommunikationsstandard sein kann. Die entwickelten Prototypen markieren den ersten Schritt zur Schaffung einer Umati-basierten Umgebung, die in der Industriepraxis Anwendung finden kann.

\section*{English}

In the work titled \glqq Development of an Approach for Integrating the VDMA Specification \glq OPC UA for Machinery (OPC 40 001)\grq{} in Machinery and IT Systems\grqq{}, the initial step towards implementing the Umati standard is explored. The main objective is to analyze the functionality of this standard and develop a method for visualizing data from machines using Umati. The underlying research questions are: What challenges arise in the integration of Umati? What solution approaches are most economically efficient? How can effective communication be established between machines and visualization platforms?

To address these questions, two prototypes are designed and implemented. The first prototype serves to demonstrate Umati functionality through the web dashboard provided by VDMA, while the second prototype implements a methodology for transferring data from machines to visualization platforms.

It is recognized that Umati can be a promising communication standard in an Industry 4.0 environment. The developed prototypes represent the initial step toward creating a Umati-based environment that can be applied in industrial practice.